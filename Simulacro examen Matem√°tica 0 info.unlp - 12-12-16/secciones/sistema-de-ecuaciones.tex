    \section{Sistema de Ecuaciones}
    
    \subsection{Ejercicio de examen}
    Determine las medidas originales de una habitación sabiendo que la perdida de área sufrida al achicar el ancho en $1,5m$ es de $5m^2$ y originalmente era el doble de larga que de ancha.
    
    \subsection{Respuesta}
    
    Este ejercicio se resuelve con sistema de ecuaciones. De esta forma, deberíamos identificar las ecuaciones, que son 2. 
    
    \underline{¿Qué datos me piden?}
    Determine las medidas originales de una habitación".
    ¿Cuáles son la medidas orginales de la habitación? El \textbf{largo} y \textbf{ancho}. 
    
    \begin{enumerate}
        \item 1ra ecuación: "la perdida de área sufrida al achicar el ancho en $1,5m$ es de $5m^2$". Recordá que el área de una habitación se calcula como $area=ancho.largo$
        \item 2da ecuación: "originalmente era el doble de larga que de ancha".
    \end{enumerate}
    
    

    La 1ra ecuación la entendemos como sigue:

    $$ (ancho-1,5m).largo = 5m^2 $$ (trabajamos sin la unidad de metros)
    
    $$ (ancho-1,5).largo = 5 $$
    
    $$ ancho.largo - 1,5largo = 5 $$
    
    La 2ra ecuación la entendemos como sigue:

    $$ largo = 2 ancho $$
    
    Si reemplamos la 2da ecuación $ \frac{largo}{2} = ancho $ en la 1ra ecuación, queda asi:
    
    $$ (\frac{largo}{2}).largo -1,5largo = 5 $$
    
    $$ \frac{largo^2}{2} -1,5largo = 5 $$
    
    $$ 2.largo^2 -1,5largo -5 = 0 $$
    
    Resolviendo con Baskara, siendo $a=\frac{1}{2}$, $b=-1,5$ y $c=-5$:
    
    $$ \frac{-b\pm\sqrt{b^2-4ac}}{2.a} $$
    
    $$ \frac{-(-1,5)\pm\sqrt{(-1,5)^2-4.\frac{1}{2}.(-5)}}{2.\frac{1}{2}} $$
    
    $$ 1,5\pm\sqrt{(-\frac{3}{2})^2+10} $$
    
    $$ 1,5\pm\sqrt{\frac{9}{4}+10} $$

    $$ 1,5\pm\sqrt{\frac{49}{4}} $$
    
    $$ \frac{3}{2} \pm \frac{\sqrt{49}}{2} $$
    
    $$ \frac{3}{2} \pm \frac{7}{2} $$
    
    $$ \frac{3 \pm 7 }{2}  $$
    
        
    Por lo tanto, obtenemos 2 resultados:
    
    $$largo_1 = 5 $$
    $$largo_2 = -2 $$
    
    Pero como el $largo$ de la habitación es en metros y tiene que ser un valor positivo, entonces nos quedamos con $largo_1$ \\
    
    Nos falta calcular \textbf{ancho}: \\
    
    $$ ancho = 2 .largo $$
    
    $$ ancho = 2 . 5 $$
    
    $$ ancho = 10 $$
    
    
    Por lo tanto, las medidas de la habitación son las siguiente:
    
    $$ \textbf{ancho} = 10 \textbf{m} $$  
    
    $$ \textbf{largo} = 5 \textbf{m} $$
    
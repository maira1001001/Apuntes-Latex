\section{Polinomios}
    
    \subsection{Ejercicio Examen}
    
        Hallar el polinomio $P(x)$ sabiendo que es divisible por $Q(x)= 2x^5 - 3x^2 - 2x - 1$, el cociente es $C(x)= 2x^3 + 4x$    y el resto es $R(x) = -3x^2 + 6x + 1$
        
    
    \subsection{Teorema del Algoritmo de la División para polinomios}
    
        $$P(x)=Q(x).C(x)+R(x)$$ 
        
        donde $P(x)$ es el dividendo, $Q(x)$ el divisor, , $C(x)$ el cociente y $R(x)$ el resto y verificandose además, que el grado de $R(x)$ es menor que el grado de $Q(x)$.
    
    \subsection{Respuesta}
        
        $$ P(x)=(2x^5 - 3x^2 - 2x - 1)(2x^3 + 4x) -3x^2 + 6x + 1 $$
        $$ P(x) = 2x^8-6x^5-4x^4-2x^3+8x^6-12x^3-8x^2-4x-3x^2+6x+1$$
        $$ P(x)=2x^8+8x^6-6x^5-4x^4-2x^3-11x^2+2x+1$$
        
    
    El algoritmo de la división para polinomios lo conocemos de la escuela: 
    
    Si realizamos la división $15:2$, el resultado es $7$ y el resto es $1$, es decir que podemos escribir $15=2.7+1$
    
    Los números reales se pueden pensar en los polinimios
    
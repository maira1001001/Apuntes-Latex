\section{Series y sumatorias}
Generalmente no distinguimos entre series y sumatorias, pero técnicamente la sumatoria es solamente la \textit{notación} para escribir \textbf{series}. Intuitivamente, una serie es simplemente la suma de ciertos términos. Si bien esto suena muy parecido a una sumatoria, la diferencia, entonces, es similar a la diferencia entre la sintaxis del for, que varía dependiendo del lenguaje, y el concepto del for o la iteración. La sintaxis del for sirve para darle una representación al concepto del for, de la misma forma que una sumatoria sirve para representar una serie.

Podemos considerar dos tipos de series, las finitas y las infinitas. Las que utilizamos para resolver los tiempos de ejecución pertenecen al grupo de las  \textbf{series finitas}, donde siempre sumamos una cantidad finita de términos. En este caso, podemos ver a las series simplemente como una suma de números.

Cuando la cantidad de términos a sumar es \textbf{infinita}, por ejemplo, $4+4+4+4+4+\dots = \sum_{i=1}^{\infty}{4}$, estamos lidiando con una \textbf{serie infinita}, entonces las propiedades usuales que vimos en esta sección no funcionan siempre, y no podemos ver a la sumatoria simplemente como una forma de escribir sumas de manera más sencilla. Por ejemplo, la serie $\sum_{i=1}^{\infty}{(-1)^n}= (-1) + 1 + (-1) + 1 + \dots$ no da 0, sino que está indefinida, o sea, no es \textbf{sumable}. 
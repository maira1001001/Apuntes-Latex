\section{Un poco de historia}

Érase una vez un niño alemán llamado {\bf Carl Friedrich Gauss. }
Cuando tenía diez años, en $1787$, su profesor de la escuela, enojado porque sus alumnos se portaban mal, les puso un problema matemático al pequeño Carl y a sus compañeros.
Los niños debían sumar todos los números del $1$ al $100$; 
es decir, $1+2+3+4+5+6+...+99+100$. El profesor se sentó en su silla a leer el periódico, 
confiaba en que tendría horas hasta que los niños sumaran todos los números. 
Sin embargo, el pequeño Gauss no tardó ni cinco minutos en ir hacia el profesor y darle el resultado: $5050$. {\it¿Cómo lo había hecho?} \newline
Gauss tenía que sumar lo siguiente:
$1+2+3+4+5+6+7+8+...+95+96+97+98+99+100$ \\
Se dio cuenta de que reordenar los elementos de esta suma, sumando siempre los simétricos, facilitaba enormemente las cosas:\\

\[ 
\left \{
  \begin{tabular}{ccccc}
  1 & + & 100 & = & 101  \\
  2 & + & 99 & = & 101  \\
  3 & + & 98 & = & 101  \\
  4 & + & 97 & = & 101  \\
  5 & + & 96 & = & 101  \\
  \ldots  \\
  46 & + & 55 & = & 101  \\
  47 & + & 54 & = & 101  \\
  48 & + & 53 & = & 101  \\
  49 & + & 52 & = & 101  \\
  50 & + & 51 & = & 101  \\
  \end{tabular}
\right \}\textbf{50 veces 101, es decir 50x101}
\]

De donde se deduce la fórmula de la sumatoria de los $n$ primeros números:

\[ \sum_{i=1}^{n}{i} =  \frac{n}{2}.(n+1)=\frac{n. (n+1)}{2} \]


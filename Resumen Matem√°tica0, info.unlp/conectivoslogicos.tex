\definecolor{naranja-fuente}{rgb}{245,209,146}


\definecolor{naranja-fuente}{rgb}{245,209,146}


\section{Conectivos Lógicos}
  \subsection{Negaci\'on}
  \begin{tabular}{cc}
    $p$ & $\neg p$ \\ \hline
    $V$ & $F$ \\
    $F$ &$V$ \\
  \end{tabular}
\subsection{Conjunci\'on}
  \begin{tabular}{ccc}
    $p$  & $q$ & $p \wedge q$ \\ \hline
    $V$ & $V$ & $V$ \\
    $V$ & $F$ &$F$ \\
    $F$ & $V$ & $F$ \\
    $F$ & $F$ & $F$\\
  \end{tabular}
\subsection{Disjunci\'on}
  \begin{tabular}{ccc}
    $p$  & $q$ & $p \vee q$ \\ \hline
    $V$ & $V$ & $V$ \\
    $V$ & $F$ &$V$ \\
    $F$ & $V$ & $V$ \\
    $F$ & $F$ & $F$\\
  \end{tabular}

\subsection{Condicional o implicaci\'on}
  \begin{tabular}{ccc}
    $p$  & $q$ & $p \rightarrow q$ \\ \hline
    $V$ & $V$ & $V$ \\
    $V$ & $F$ &$F$ \\
    $F$ & $V$ & $V$ \\
    $F$ & $F$ & $V$\\
  \end{tabular} \\
%\newline
$p \rightarrow q$ se lee:
  \begin{itemize}
   \item $p$ implica $q$
    \item si $p$, entonces $q$
     \item $p$ s\'olo si $q$
      \item $q$ si $p$
      \item $p$ es condici\'on suficiente para $q$
      \item $q$ es condici\'on necesaria para $p$
  \end{itemize}
%\subsection{Rec\'iproco del implicador}
%  \begin{tabular}{cccc}
%    $p$  & $q$ & $p \rightarrow q$ & $p \leftarrow q$\\ \hline
%    $V$ & $V$ & $V$ & $V$\\
%    $V$ & $F$ &$F$  & $V$\\
%    $F$ & $V$ & $V$ & $F$\\
%    $F$ & $F$ & $V$ & $V$\\
%  \end{tabular} \\
%\subsection{Contrarrec\'iproco del implicador}
%  \begin{tabular}{cccccc}
%    $p$  & $q$ & $p \rightarrow q$ & $\neg q$ & $\neg p$ &$ \neg q \rightarrow \neg p$\\ \hline
%    $V$  & $V$ & $V$ 		   &  $F$     &	$F$	 &  $V$ \\
%    $V$  & $F$ & $F$ 		   &  $V$     &	$F$	 &  $F$\\
%    $F$  & $V$ & $V$ 		   &  $F$     &	$V$	 &  $V$ \\	
%    $F$  & $F$ & $V$ 		   &  $V$     &	$V$	 &  $V$ \\
%  \end{tabular} \\
\subsection{Rec\'iproco de la implicaci\'on}
    El rec\'iproco de $p \rightarrow q$  es $q \rightarrow p$\\ 
    Tienen valores diferentes\\
 
\subsection{Contrarrec\'iproco de la implicaci\'on }
     El contrarrec\'iproco de $p \rightarrow q$  es $\neg q \rightarrow \neg p$\\ 
     Tienen los mismos valores \\

  \subsection{Bicondicional}

  \begin{tabular}{ccc}
    $p$  & $q$ & $p \leftrightarrow q$ \\ \hline
    $V$ & $V$ & $V$ \\
    $V$ & $F$ &$F$ \\
    $F$ & $V$ & $F$ \\
    $F$ & $F$ & $V$\\
  \end{tabular} \\

  $p \leftrightarrow q$ se lee:
  \begin{itemize}
   \item $p$ si s\'olo si $q$
    \item $p$ es necesario y suficiente para $q$
  \end{itemize}

  \subsection{Tautolog\'ia}
    \subsubsection{Doble negaci\'on}
      $\neg \neg p \Leftrightarrow p$ \\
    
    \subsubsection{Leyes conmutativas}
      \renewcommand{\labelenumi}{\alph{enumi})}
	\begin{enumerate}
      \item $(p \wedge q) \Leftrightarrow (q \wedge p)$\\

      \item $(p \vee q) \Leftrightarrow (q \vee p)$\\
	\end{enumerate}
	

    \subsubsection{Leyes asociativas}
     \renewcommand{\labelenumi}{\alph{enumi})}
      \begin{enumerate}
      \item $[(p \wedge q) \wedge r)] \Leftrightarrow [p \wedge (q \wedge r)]$

      \item $[(p \vee q) \vee r)] \Leftrightarrow [p \vee (q \vee r)]$
      \end{enumerate}
      
      
    \subsubsection{Leyes distributivas}

      \renewcommand{\labelenumi}{\alph{enumi})}
	\begin{enumerate}

      \item $[p \vee (q \wedge r)] \Leftrightarrow [(p \vee q) \wedge (p \vee r)]$

      \item $[p \wedge (q \vee r)] \Leftrightarrow [(p \wedge q) \vee (p \wedge r)]$
      \end{enumerate}
      

    \subsubsection{Leyes de idempotencia}
      \renewcommand{\labelenumi}{\alph{enumi})}
      \begin{enumerate}
	\item $(p \vee p) \Leftrightarrow p$

	\item $(p \wedge p) \Leftrightarrow p$
      \end{enumerate}
      


    \subsubsection{Leyes  de De Morgan}
      \renewcommand{\labelenumi}{\alph{enumi})}
	\begin{enumerate}

      \item $\neg (p \vee q) \Leftrightarrow (\neg p \wedge \neg q)$

      \item $\neg (p \wedge q) \Leftrightarrow (\neg p \vee \neg q)$

	\end{enumerate}
	
    \subsubsection{Implicaci\'on}
      \renewcommand{\labelenumi}{\alph{enumi})}
	\begin{enumerate}

	  \item $(p \rightarrow q) \Leftrightarrow (\neg p \vee q)$
	  
	  \item $[(p \rightarrow r) \wedge (q \rightarrow r )] \Leftrightarrow [(p \wedge q) \rightarrow r )]$

	  \item $[(p \rightarrow q) \wedge (p \rightarrow r )] \Leftrightarrow [(p \rightarrow  (q \wedge r) )]$
	\end{enumerate}
	

    \subsubsection{Equivalencia}

    $(p \leftrightarrow q)\Leftrightarrow[(p \rightarrow q) \wedge (q \rightarrow p)]$

    \subsubsection{Adici\'on}

    $p \Rightarrow (p \vee q)$

    \subsubsection{Simplificaci\'on}

    $(p \wedge q) \Rightarrow p$

    \subsubsection{Absurdo}

    $(p \rightarrow F)\Rightarrow \neg p$

    \subsubsection{Modus ponens}

    $[p \wedge (p \rightarrow q)] \Rightarrow q$

    \subsubsection{Modus tollens}

    $[(p \rightarrow q) \wedge \neg q] \Rightarrow \neg p$
  
    \subsubsection{Transitividad del $\leftrightarrow $}

    $[(p \leftrightarrow q) \wedge (q \leftrightarrow r)] \Rightarrow (p \leftrightarrow r)$

    \subsubsection{Transitividad del $\rightarrow $}

    $[(p \rightarrow q) \wedge (q \rightarrow r)] \Rightarrow (p \rightarrow r)$

    \subsubsection{Dilemas constructivos}
      \renewcommand{\labelenumi}{\alph{enumi})}
	\begin{enumerate}
	  \item $[(p \rightarrow q) \vee (r \rightarrow s)] \Rightarrow [(p \vee r) \rightarrow (q \vee s)]$

	  \item $[(p \rightarrow q) \wedge (r \rightarrow s)] \Rightarrow [(p \wedge r) \rightarrow (q \wedge s)]$
	\end{enumerate}

  \subsection{Contradicci\'on}
	\begin{tabular}{ccc}
	$p$ & $\neg p$ & $p \wedge \neg p$ \\ \hline
	$F$ & $V$ & $F$ \\
	$V$ & $F$ &$F$ \\
      \end{tabular} \\
      Ej: ``La computadoras es negra y la computadora no es negra''
\subsection{Equivalencia l\'ogica}
Cuando los resultados de dos proposiciones tienen los mismos valores de verdad, se indica como $p \Leftrightarrow q$

\section{Operadores Universal y Existencial}

  \subsection{Universal}
    $\forall x : p(x) $ \\
    Puede leerse :
    \begin{itemize}
     \item Para todo $x, p(x)$
      \item Para cada $x$
       \item Para cualquier $x$
        \item Para $x$ arbitrario
    \end{itemize}

    
  \subsection{Existencial}
    $\exists x: p(x)$  \\
    Puede leerse :
    \begin{itemize}
    \item Existe un $x$ tal que  $p(x)$
      \item Para alg\'un $x$, $p(x)$
      \item Existe, al menos, un $x$ tal que $p(x)$
    \end{itemize}

  \subsection{Equivalencias}
  Sea $A$ el universo,
    \begin{itemize}
    \item $\neg \forall x \in A \thickspace p(x) \qquad \Leftrightarrow \qquad \exists x \in A: \thickspace \neg p(x) $ 
     \item $\neg \exists x \in A: \thickspace p(x) \qquad \Leftrightarrow \qquad \forall x \in A \thickspace \neg p(x)$
    \end{itemize}
    
    

  \subsection{Alcance de un operador}
El alcance se indica con par\'entesis\\
$(\exists x): \thickspace x \medspace es \medspace verde \thickspace  \wedge \thickspace x \medspace es \medspace rojo$ \\
$(\exists x): (\thickspace x \medspace es \medspace verde \thickspace \wedge \thickspace x \medspace es \medspace		 rojo)$ 

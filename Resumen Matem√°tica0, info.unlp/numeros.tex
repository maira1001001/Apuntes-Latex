\section{Números}
%------------------------numeros naturales----------------------------------------
   \subsection{N\'umeros $N$}
   \subsubsection{Te comentamos sobre los naturales}
      
      \begin{itemize}
	 \item Que un número natural va después del otro
	 \item Que dentro de dos números naturales consecutivos no puede haber otro
	 \item Que son infinitos
      \end{itemize}

   \subsubsection{Propiedades}
      \begin{itemize}
       \item $0 \in \mathbb{N} $
      \item $n \in \mathbb{N} \Rightarrow n+1 \in \mathbb{N}$
      \item si $n \in \mathbb{N}$, entonces $n-1 \in \mathbb{N}$, cuando $n \neq 0$
      \item sea $M$ subconjunto de $\mathbb{N}$ con las siguiente propiedades
	 \begin{itemize}
	 \item $0 \in M$
	 \item $n \in M \Rightarrow n+1 \in M \Rightarrow M=\mathbb{N} $ 
	 \end{itemize}
      \end{itemize}

   \subsection{Propiedades Algebraicas}
   Las operaciones de suma $(+)$ con $N$: 
      \begin{itemize}
	 \item Ley de Cierre: $\forall a, b \in N \Rightarrow a+b \in N$ 
	 \item Asociativo: $\forall a,b,c \in N: a+(b+c)=(a+b)+c $
	 \item Conmutativo: $\forall a,b \in N: a+b=b+a$
	 \item Elemento neutro: $\exists 0 \in N \thickspace tal \thickspace que \thickspace \forall a \in N: a+0=a$
       \end{itemize}
      La suma de dos n\'umeros $N$ es cerrado, es decir, es un $N$. Sin embargo, para $n < m: \quad n-m \notin N$ \newline
      Las operaciones de multiplicaci\'on $(\times)$ con $N$:
      \begin{itemize}
	 \item Ley de Cierre: $\forall a, b \in N \Rightarrow a \times b \in N$
	 \item Asociativo: $\forall a,b,c \in N: a \times(b \times c)=(a \times b) \times c $
	 \item Conmutativo: $\forall a,b \in N: a \times b = b \times a $
	 \item Elemento neutro: $\exists 1 \in N \thickspace tal \thickspace que \thickspace \forall a \in N: a \times 1= a$
	 \item Distributivo la $\times$ con respecto a la $+$: $\forall a,b,c \in N: a \times (b+c)= (a \times b) + (a \times c)$
	 \item No hay divisores de cero:  $ \forall a,b \in N:$ Si $a \times b = 0 \Rightarrow a=0 \vee b=0$ 
      \end{itemize}

%------------------------numeros enteros----------------------------------------
\subsection{Números $\mathbb{Z}$}
    \subsubsection{Propiedades Algebraicas}
      Las operaciones de suma $(+)$ con $\mathbb{Z}$: 
      \begin{itemize}
	 \item Ley de Cierre: $\forall a, b \in \mathbb{Z} \Rightarrow a+b \in \mathbb{Z}$ 
	 \item Asociativo: $\forall a,b,c \in \mathbb{Z}: a+(b+c)=(a+b)+c $
	 \item Conmutativo: $\forall a,b \in \mathbb{Z}: a+b=b+a$
	 \item Elemento neutro: $\exists 0 \in \mathbb{Z} \thickspace tal \thickspace que \thickspace \forall a \in \mathbb{Z}: a+0=a$
	 \item Elemento Opuesto: $\forall a \in \mathbb{Z}, \exists -a \in \mathbb{Z}: a+(-a)=0$
       \end{itemize}
      Las operaciones de multiplicaci\'on $(\times)$ con $\mathbb{Z}$:
      \begin{itemize}
	 \item Ley de Cierre: $\forall a, b \in \mathbb{Z} \Rightarrow a \times b \in \mathbb{Z}$
	 \item Asociativo: $\forall a,b,c \in \mathbb{Z}: a \times(b \times c)=(a \times b) \times c $
	 \item Conmutativo: $\forall a,b \in \mathbb{Z}: a \times b = b \times a $
	 \item Elemento neutro: $\exists 1 \in \mathbb{Z} \thickspace tal \thickspace que \thickspace \forall a \in \mathbb{Z}: a \times 1= a$
	 \item Distributivo la $\times$ con respecto a la $+$: $\forall a,b,c \in \mathbb{Z}: a \times (b+c)= (a \times b) + (a \times c)$
	 \item No hay divisores de cero:  $ \forall a,b \in \mathbb{Z}:$ Si $a \times b = 0 \Rightarrow a=0 \vee b=0$ 
      \end{itemize}
      \subsubsection{N\'umeros Pares}
      Un n\'umero entero $n$ es par $\Leftrightarrow \exists k \in \mathbb{Z}: n=2k$ 

      \subsubsection{N\'umeros Impares}
      Un n\'umero entero $n$ es impar $\Leftrightarrow \exists k \in \mathbb{Z}: n=2k+1$ 

      \subsubsection{Divisibilidad}
      Sean $a,b \in \mathbb{Z}$, decimos que $a$ divide a $b$ si $\exists c \in \mathbb{Z}:  b=ac$ \newline
      Se suele expresar de la forma $a|b$, y se lee ``$a$ divide $b$'', ``$a$ es un divisor de $b$''o tambi\'en  ``$b$ es m\'ultiplo de $a$ ''

   \subsubsection{Ley de Monoton\'ia}
   Si $a, b, c\in \mathbb{Z} \wedge a \leq b \Rightarrow a+c \leq b+c$

   \subsubsection{Propiedades del orden de los $\mathbb{Z}$}
   \begin{itemize}
    \item Sea $ a,b,c \in \mathbb{Z}$, Si $ a \leq b \wedge c \leq d \Rightarrow a+c \leq b+d$
     \item Sea $ a,b,c \in \mathbb{Z}$, Si $ a \leq b \wedge 0 < c \Rightarrow ac < bc$
   \end{itemize}

   \subsubsection{Primos}
   Un n\'umero $p$ es primo, si tiene exactamente 4 divisores: $1, -1, p$ y $-p$

%\newpage
%------------------------numeros racionales----------------------------------------
   \subsection{N\'umeros $\mathbb{Q}$}
      \subsubsection{Definici\'on}
      Se llama número racional a todo número que puede representarse como el cociente de dos números enteros, es decir, una fracción común a/b con numerador $a$ y denominador $b$ distinto de cero. \newline
      Esto es equivalente a escribir formalmente:
      \begin{displaymath}
	 \mathbb{Q}= \left\{ \frac{a}{b}: a,b \in \mathbb{Q} \wedge b \neq 0 \right\} 
      \end{displaymath}
      \subsubsection{Suma en $\mathbb{Q}$}
      Dados $\frac{a}{b}$ y $\frac{c}{d} \in \mathbb{Q}$, se define la suma: \[\frac{a}{b} + \frac{c}{d}= \frac{ad+bc}{bd} \]
      \subsubsection{Multiplicaci\'on en $\mathbb{Q}$}
      Dados $\frac{a}{b}$ y $\frac{c}{d} \in \mathbb{Q}$, se define la multiplicaci\'on: \[ \frac{a}{b} \times \frac{c}{d}= \frac{ac}{bd} \]
      \subsubsection{N\'umeros equivalentes}
      Sean $\frac{a}{b}$ y $\frac{c}{d} \in  \mathbb{Q}$ son equivalentes si y s\'olo si $a\cdot d=b \cdot c$
      \subsection{Propiedades Algebraicas}
      Las operaciones de suma $(+)$ con $\mathbb{Q}$: 
      \begin{itemize}
	 \item Ley de Cierre: $\forall p,q \in \mathbb{Q} \Rightarrow p+q \in \mathbb{Q}$ 
	 \item Asociativo: $\forall p,q,r \in \mathbb{Q}: p+(q+r)=(p+q)+r $
	 \item Conmutativo: $\forall p,q \in \mathbb{Q}: p+q=q+p$
	 \item Elemento neutro: $\exists 0 \in \mathbb{Q} \thickspace tal \thickspace que \thickspace  \forall p \in \mathbb{Q}: p+0=p$
	 \item Elemento Opuesto: $\forall p \in \mathbb{Q}, \exists -p \in \mathbb{Q}: p+(-p)=0$
       \end{itemize}
      Las operaciones de multiplicaci\'on $(\times)$ con $\mathbb{Q}$:
      \begin{itemize}
	 \item Ley de Cierre: $\forall p,q \in \mathbb{Q} \Rightarrow p \times q \in \mathbb{Q}$
	 \item Asociativo: $\forall p,q,r \in \mathbb{Q}: p \times(q \times r)=(p \times q) \times r $
	 \item Conmutativo: $\forall p,q \in \mathbb{Q}: p \times q = q \times p $
	 \item Elemento neutro: $\exists 1 \in \mathbb{Q} \thickspace tal \thickspace que \thickspace \forall p \in \mathbb{Q}: p \times 1= p$
	 \item Inverso multiplicativo: $\forall p \in \mathbb{Q}, p \neq 0, \exists p^{-1} \in \mathbb{Q} : p \times p^{-1} = 1$
	 \item Distributividad de la $(\times)$ en la $(+)$: Si $p,q,r \in \mathbb{Q}$ entonces $p \times (q + r)= (p \times q) + (p \times r)$
      \end{itemize}
      \subsubsection{Orden en $\mathbb{Q}$}
      Dados $\frac{a}{b}$ y  $\frac{c}{d} \in \mathbb{Q}$, se dice que:
      \begin{itemize}
       \item $\frac{a}{b}$ es menor o igual que $\frac{c}{d}$ y se anota $\frac{a}{b} \leq \frac{c}{d} \Leftrightarrow ad \leq bc$ 
       \item $\frac{a}{b}$ es mayor o igual que $\frac{c}{d}$ y se anota $\frac{a}{b} \geq \frac{c}{d} \Leftrightarrow ad \geq bc$ 
      \end{itemize}
       \subsubsection{$\mathbb{Q}$ es denso con la relaci\'on $\leq$}
      Entre dos racionales distintos, siempre existe otro: \\ Sean \[\frac{a}{b}\quad y \quad \frac{c}{d} \in Q \Rightarrow   \frac{a}{b} \leq \frac{a+b}{c+d} \leq \frac{c}{d} \]

  %-----------------------------------------numeros Irracionales-------------------------------------------------------
   \subsection{N\'umeros I}
   Los n\'umeros irracionales son aquellos que no pueden expresarse como cociente de dos n\'umeros enteros $\frac{a}{b}$, con $n \neq 0$ \newline Entre los irracionales m\'as conocidos est\'an $\sqrt{2}$, $\sqrt{3}$, $\pi$

%---------------------------------------------numeros Reales-------------------------------------------------------------
    \subsection{N\'umeros R}
    \subsubsection{Propiedades Algebraicas}
      Las operaciones de suma $(+)$ con $R$: 
      \begin{itemize}
	 \item Ley de Cierre: $\forall a, b \in R \Rightarrow a+b \in R$ 
	 \item Asociativo: $\forall a,b,c \in R: a+(b+c)=(a+b)+c $
	 \item Conmutativo: $\forall a,b \in R: a+b=b+a$
	 \item Elemento neutro: $\exists 0 \in R \thickspace tal \thickspace que \thickspace \forall a \in R: a+0=a$
	 \item Elemento Opuesto: $\forall a \in R, \exists -a \in R: a+(-a)=0$
       \end{itemize}
      Las operaciones de multiplicaci\'on $(\times)$ con $R$:
      \begin{itemize}
	 \item Ley de Cierre: $\forall a, b \in R \Rightarrow a \times b \in R$
	 \item Asociativo: $\forall a,b,c \in R: a \times(b \times c)=(a \times b) \times c $
	 \item Conmutativo: $\forall a,b \in R: a \times b = b \times a $
	 \item Elemento neutro: $\exists 1 \in R \thickspace tal \thickspace que \thickspace \forall a \in R: a \times 1 = a$
	 \item Inverso multiplicativo: $\forall a \in R, a \neq 0, \exists a^{-1} \in R : a \times a^{-1} =  1$
	 \item Distributivo la $\times$ con respecto a la $+$: $\forall a,b,c \in R: a \times (b+c)= (a \times b) + (a \times c)$
      \end{itemize}
    \subsubsection{Propiedades de Orden}
    \begin{itemize}
     \item  $\forall a \in R: $ $a<0 \Rightarrow a=0 \vee a>0$ 
      \item $a,b \in R$, si $ a>0 \wedge b>0 \Rightarrow a \cdot b >0$
      \item $\forall a,b \in R$, si $a<b \Rightarrow a-b <0$
    \end{itemize}
    \subsubsection{Potencia de un número R y exponente entero}
    Sea $a \in R, n \in N, n \neq 0$ entonces $a^n=\underbrace{aa\ldots a}_{n \quad veces} $ \newline
    Recordar: como $R$ tiene elemento inverso, entonces $a^n a^{-n}=1 \Rightarrow a^{-n}=\frac{1}{a^n}$, siempre que $a \neq 0$.
    \subsubsection{Propiedades de las potencias}
    Sean $a,b \in R$, y $n \in Z$
    \begin{itemize}
     \item $(a.b)^n=a^n b^n$
     \item $(\frac{a}{b})^n=\frac{a^n}{b^n}$
     \item $ a^n . a^m=a^{n+m}$
     \item $ a^n : a^m=a^{n-m} $, y $a \neq 0$
     \item $ (a^n)^m= a^{n.m}=a^{m.n}=(a^m)^n$
     \item \textbf{OJO}: $(a \pm b)^n \neq a^n\pm b^n$ 
    \end{itemize}
    \subsubsection{Radicación}
    $b \in R$, $n \in Z$ y $n>1$, $\exists c$:
    \[ c^n=b \Leftrightarrow c=\sqrt[n]{b}\]
    Observar que si $b\in \mathbb{R} \quad$es:
    \begin{itemize}
     \item $b$ es impar: $\sqrt[n]{b}$ existe
     \item $b$ es par: $\sqrt[n]{b}$ existe sólo si $b \geq 0$
    \end{itemize}
    \subsubsection{Propiedades de la radicación}
    Sean $a,b \in R$, y $n \in Z$
    \begin{itemize}
     \item $\sqrt[n]{a.b}=\sqrt[n]{a}. \sqrt[n]{b}$
     \item $\sqrt[n]{ \frac{a}{b}}=\frac{ \sqrt[n]{a} }{ \sqrt[n]{b}}$
     \item $ \sqrt[n]{a} . \sqrt[m]{a}=\sqrt[n+m]{a}$
     \item $ \sqrt[n]{a} : \sqrt[m]{a}=\sqrt[n-m]{a} \quad$, y $a \neq 0$
     \item $ \sqrt[m]{ \sqrt[n]{a} }= \sqrt[n.m]{a}= \sqrt[m.n]{a}=\sqrt[n]{\sqrt[m]{a}}$
     \item $\textbf{¡¡OJO!!}: \sqrt[n]{a \pm b} \neq \sqrt[n]{a}\pm \sqrt[n]{b}$ 

    \end{itemize}
    
    \subsubsection{Raíz Aritmética}
    \begin{itemize}
     \item Si $n$ es par: $ \sqrt[n]{a^n} = a$
     \item Si $n$ es impar:$ \sqrt[n]{a^n} = |a|$
    \end{itemize}

    \subsubsection{Racionalización de Denominadores}
    Tenemos dos casos:
    \begin{itemize}
     \item $\frac{A}{ \sqrt{a} \pm \sqrt{b}} = \frac{A}{ \sqrt{a} \pm \sqrt{b}} . \frac{\sqrt{a} \mp \sqrt{b}}{\sqrt{a} \mp \sqrt{b}}$
     \item $\frac{A}{ \sqrt{a} \pm c} = \frac{A}{ \sqrt{a} \pm c} . \frac{\sqrt{a} \mp c}{\sqrt{a} \mp c}$
    \end{itemize}
    
    \subsubsection{Potencias de Exponente Racional}
    Sean $n,m \in Z$, y $n \neq 0$: \newline
    \begin{itemize}
     \item  $a^{\frac{m}{n}}= \sqrt[n]{a^{m}}= (\sqrt[n]{a})^m $
     \item $a^{m+0}=a^m.a^0=a^m$
     \item  $a^{-\frac{m}{n}}= \frac{1}{a^{\frac{m}{n}}}$
    \end{itemize}
  
\usepackage{multicol}
\usepackage[landscape]{geometry}
\usepackage[utf8]{inputenc} % informa el sistema usado de codificacion
% alternativa puede ser \usepackage[latin1]{inputenc}, depende del sistema
\usepackage[T1]{fontenc} % asegura los tipos de letras en pdf, p.ej. ñ
\usepackage[spanish]{babel} % Sirve para traducir comandos, p.ej. 
% en vez de Table of contents sera Índice
% También para hacer separación de palabras en forma correcta.
\usepackage{amssymb}
\usepackage{amsmath} 
\usepackage{amsfonts}
\usepackage{amstext}
\usepackage{subfig} %para dos tablas side by side en una columna.
\usepackage{graphics}
\usepackage{wrapfig}
\usepackage{cancel}
\usepackage{tikz} % paquete para dibujar dentro de latex
% para cabecera de p\'agina
\usepackage{ctable}
\usepackage{fontawesome}% Provide icons
\usepackage{fancyhdr} % headers and footers
\usepackage{colortbl}% color a la tabla
\usepackage{draftwatermark} % marca de agua

